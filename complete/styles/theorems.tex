%%%%%%%%%%%%%%%%%%%%%%%%%%%%%%%%%%%%%%%%%%%%%%%%%%%%%%%%%%%%%%%%%%%%%%%%%%%%%%%%%%%%%%%%%%%%%%
%
% This is where one would tell \LaTeX{} how to format Theorems, Definitions, etc. and also
% indicate the environment names. You need the amsthm package (loaded in the woosterthesis %class) in order for these commands to work.
%
%%%%%%%%%%%%%%%%%%%%%%%%%%%%%%%%%%%%%%%%%%%%%%%%%%%%%%%%%%%%%%%%%%%%%%%%%%%%%%%%%%%%%%%%%%%%%%

% an example of defining your own theoremstyle
%\newtheoremstyle{break}% name
%  {\topsep}%      Space above
%  {\topsep}%      Space below
%  {\itshape}%         Body font
%  {}%         Indent amount (empty = no indent, \parindent = para indent)
%  {\bfseries}% Thm head font
%  {.}%        Punctuation after thm head
%  {\newline}%     Space after thm head: " " = normal interword space;
%        %       \newline = linebreak
%  {}%         Thm head spec (can be left empty, meaning `normal')
\newtheoremstyle{scthm}{\topsep}{\topsep}{\itshape}{}{\bfseries\scshape}{}{ }{}% small cap font for the heading
\newtheoremstyle{itdefn}{\topsep}{\topsep}{\itshape}{}{\bfseries}{.}{ }{}% italic definitions
\newtheoremstyle{scdefn}{\topsep}{\topsep}{\itshape}{}{}{}{ }{\thmname{\textbf{#1}}\thmnumber{ \textbf{#2}}\thmnote{ \scshape #3:}}% small cap headings and italic text.

\theoremstyle{break}% this theoremstyle will put the text of the theorem on a new line.
\newtheorem{thm}{Theorem}[chapter]%number theorems within chapters 
%\newtheorem{cor}[thm]{Corollary}%by using [thm] we are numbering these environments with the theorems.
\newtheorem{cor}{Corollary}[chapter]%number corollaries within chapters .
%\newtheorem{lem}[thm]{Lemma}
\newtheorem{lem}{Lemma}[chapter]
%\newtheorem{prop}[thm]{Proposition}
\newtheorem{prop}{Proposition}[chapter]

\theoremstyle{scdefn}
%\newtheorem{defn}[thm]{Definition}
\newtheorem{defn}{Definition}[chapter]
\theoremstyle{remark}
%\newtheorem{rem}[thm]{Remark}
\newtheorem{rem}{Remark}[chapter]
\renewcommand{\therem}{}
%\newtheorem{ex}[thm]{Example}
\newtheorem{ex}{Example}[chapter]

\theoremstyle{plain}
%\newtheorem{note}[thm]{Notation}
\newtheorem{note}{Notation}[chapter]
\renewcommand{\thenote}{}
%\newtheorem{nts}[thm]{Note to self}%use to remind yourself of things yet to do
\newtheorem{nts}{Note to self}[chapter]
\renewcommand{\thents}{}
%\newtheorem{terminology}[thm]{Terminology}
\newtheorem{terminology}{Terminology}[chapter]
\renewcommand{\theterminology}{}

\theoremstyle{itdefn}
\newtheorem{bdefn}{Definition}[chapter]
\newsavebox{\fmbox} 
\newenvironment{boxeddefn}[2] 
{\begin{lrbox}{\fmbox}\begin{minipage}{0.9 \linewidth }\begin{singlespace}\begin{bdefn}[{#1}]\label{#2}\vspace{0.2cm}} 
{\end{bdefn}\end{singlespace}\end{minipage}\end{lrbox}\fbox{\usebox{\fmbox}}}