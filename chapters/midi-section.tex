\section[MIDI: An Introduction]{MIDI}\label{midi}

\subsection[What is MIDI?]{What is MIDI?}\label{what-midi}

The Musical Instrument Digital Interface (commonly known as MIDI) is a communication protocol that allows musicians to play an electronic keyboard--or another MIDI-compatible device--and achieve the sound of another. Any MIDI device can send or receive MIDI messages, including a synthesizer. A synthesizer can be considered a MIDI receiver, as it can receive MIDI messages from a MIDI controller--or the device sending MIDI messages. The most common type of MIDI controller is the electronic keyboard with piano keys.

\subsection[How does MIDI work?]{How does MIDI work?}\label{how-midi}

A MIDI receiver will receive MIDI messages that contain information about musical notes or events. These MIDI messages, or commands, are communicated through one of MIDI's 16 channels, numbered one through sixteen. These musical events include information about a note's notation, pitch, velocity\footnote{This is typically a note's volume or loudness.}, vibrato, panning information\footnote{Whether a note is panned to the right or the left of stereo.}, and clock signals\footnote{These set the tempo of a note.}, among others.\footnote{Further reading can be found in Romano 2003, which discusses several other important MIDI commands, such as note on, note off, and aftertouch.} MIDI itself does not produce sound or music, but the instructions for a MIDI receiver include which sounds to play. These instructions are given in binary, combined into byte-sized commands. The protocol itself provides support for 128 notes, which ranges from C five octaves below middle C (or $C_0$) up to $G_{10}$ (ten octaves higher).