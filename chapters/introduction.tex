%!TEX root = ../username.tex
\chapter{Introduction}\label{intro}
So why would you want to use \lt instead of Microsoft Word\texttrademark? I can think of several reasons. The main one for this author is that \lt takes care of all of the numbering automatically. This means that if you decide to rearrange material in your IS, you do not have to worry about renumbering or references. This makes it very easy to play around with the structure of your thesis. The second reason is that it is ultimately faster than Word\texttrademark. How? Well, after a week or so of using \lt you will begin to remember the commands that you use frequently and won't have to use the \lt pallet in TeXShop or TeXnicCenter. So you can just type everything including the mathematics, where with \msw you would have to use the Equation Editor.

I have also tried to make things more efficient by organizing the example folder as follows. There is a \texttt{username.tex} file which you will want to rename using your username and which is what you will enter all of the information about your IS into. \texttt{username.tex} also has explanations about other files that you might need to edit. In addition there are folders for chapters, appendices, styles, and figures. This structure is there to try and reduce file clutter and to help you stay organized. There should also be a .bib file which you can use as a model for your own .bib file. The .bib file has your bibliographic information.

\lt is really easy to learn. For an average IS, the author will only need to learn a handful of commands. For this small bit of effort, you get a tremendous amount of flexibility and a very beautiful document. The following chapters will introduce some of the common things a student might need to do in a thesis.

\section{What is in \texttt{username.tex}}
Before we move on let's talk a little bit about what is at the beginning of \verb|username.tex|. The file starts with 
\verb|\documentclass{woosterthesis}|, which must be at the beginning of every IS. In the brackets are options for the woosterthesis class. The options are the same as for the \verb|book| class with some additional options 
\verb|abstractonly|\index{woosterthesis options!abstractonly},
\verb|acs|\index{woosterthesis options!acs},
\verb|alltt|\index{woosterthesis options!alltt},
\verb|apa|\index{woosterthesis options!apa},  
\verb|blacklinks|\index{woosterthesis options!blacklinks},
\verb|chicago|\index{woosterthesis options!chicago},
\verb|code|\index{woosterthesis options!code},
\verb|colophon|\index{woosterthesis options!colophon},
\verb|dropcaps|\index{woosterthesis options!dropcaps},
\verb|euler|\index{woosterthesis options!euler},
\verb|foreignlanguage|\index{woosterthesis options!foreignlanguage},
\verb|guass|\index{woosterthesis options!guass}, 
\verb|index|\index{woosterthesis options!index},
\verb|kaukecopyright|\index{woosterthesis options!kaukecopyright},
\verb|maple|\index{woosterthesis options!maple},
\verb|mla|\index{woosterthesis options!mla},
\verb|palatino|\index{woosterthesis options!palatino},
\verb|picins|\index{woosterthesis options!picins},
\verb|tikz|\index{woosterthesis options!tikz},
\verb|verbatim|\index{woosterthesis options!verbatim},
\verb|wblack|\index{woosterthesis options!wblack},
and \verb|woostercopyright|\index{woosterthesis options!woostercopyright}. If no options are specified then the class default options \texttt{letterpaper}, \texttt{12pt}, \texttt{oneside} , \texttt{onecolumn} , \texttt{final}, and \texttt{openany} are used.

The \verb|abstractonly| option will allow you to print just the Abstract. The \verb|acs| option implements the American Chemical Society citation and reference style. The \verb|alltt| option loads the \ip{alltt} package for using typewriter type in various ways and the \verb|apa| option implements the APA citation and reference style. The \verb|blacklinks| option will make the hyperlinks in the PDF version of the thesis black and suitable for printing; normally the links are colored to provide visual clues to the reader. The \verb|chicago| option implements Chicago style citation and references. The \verb|code| option will use \ip{listings} style to format program code examples. The \verb|colophon| option will include a colophon which is a section that describes the fonts and other settings used to produce the manuscript. \verb|dropcaps| loads the \ip{lettrine} package for doing dropped capitals and the \verb|euler| and \verb|guass| options load the \ip{woofncychap} package with the named option which will change the look of chapter headings. The \verb|foreignlanguage| option will load the \ip{csquotes} package and either the \ip{polyglossia} or \ip{babel} package depending on if \xt is being used to allow the input and formatting of sections of text in a foreign language. The \verb|index| option will allow the \ip{makeidx} package to be loaded so that if you have index entries they will be added to an index (this reqires additional steps).  The \verb|kaukecopyright| option will put the Kauke Hall symbol with the pre 2021 word mark on the copyright page.  The \verb|maple| option will load the Maple package for including Maple code. The \verb|mla| option implements the MLA citation and reference style. The \verb|palatino| option will use the \ip{pxfonts} package which uses the Palatino fonts. The \verb|picins| option will use the \ip{wrapfig} package to allow text to wrap around images and \verb|verbatim| allows one to set verbatim what is entered. The \verb|tikz| option loads the \ip{Ti\emph{k}Z} package enabling users to draw figures in their \lt document. The \verb|wblack| option includes an opaque Wooster "W" (as of 8/2021) in the background of the title page instead of the default opaque Kauke Hall image and the \verb|woostercopyright| option includes a copyright notice with the new (as of 8/2021) Wooster word mark. Adding or deleting options from the comma separated list will change the appearance of the document and some options should only be used after consulting your advisor. Now let's move on to some other things that you'll need to deal with: figures, pictures, and tables.