\section[SuperCollider: A Domain-Specific Language]{SuperCollider: A Domain-Specific Language}\label{supercollider}

SuperCollider is a domain-specific programming language (DSL) and environment created in 1996 by James McCartney. A domain-specific language is a programming language with a higher level of abstraction, and optimized to solve a specific class of problems. It will use the concepts from the specific field it is built and designed for. A domain-specific language will differ from a ''normal'' programming language in that it is usually less complex than a general-purpose language such as Java. Most often, these are built to be used by non-developers who are familiar with and experts in the domain that the DSL addresses.\footnote{More information can be found in JetBrains' article about DSLs  \url{https://www.jetbrains.com/mps/concepts/domain-specific-languages}} It was built for real-time audio synthesis and algorithmic composition\footnote{This project does not focus on algorithmic composition, which put most simply is a method of composing music using an algorithm or multiple algorithms.}. Since 1996, SuperCollider has evolved into an environment that is actively used and developed upon by both scientists and artists working with sound. In 2002, it was released as open-source software under the GNU General Public License.

The SuperCollider environment itself is made up of two parts: the server \textit{scsynth} and the client \textit{sclang} \cite{McCartney_2002}. The two parts complete complementary tasks, with the \texttt{scsynth} server performing sound synthesis, and the \texttt{sclang} client writes, compiles, and executes programs which define and control sound synthesis. The SuperCollider application itself is unable to perform sound synthesis, sending commands to the \texttt{scsynth} server. The server responds to requests from the client to synthesize and produce sounds \cite{McCartney_2021}. 