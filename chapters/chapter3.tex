%!TEX root = ../main.tex
\chapter{Working with bibliographies and indicies}\label{bibind}
I would highly recommend that you use Bib\LaTeX{} for your bibliography, and that is what this class uses as of August 2022. Bib\LaTeX{} supports foreign languages and UTF-8 and provides several styles for formatting references (ACS, Chicago, APA, Turabian). For many people it also is easier to customize than Bib\TeX{} with natbib, should you need to customize the formatting of references. Bib\LaTeX{} also can make use of the Biber reference processing application which provides support for foreign languages and more sophisticated sorting options than Bib\TeX{}. You still process a special .bib file. The .bib file is where you enter your bibliographic information. Sample entries look something like
\begin{singlespace}\small
\begin{verbatim}
@article{feu02,
author=		{Thomas Feuerstack},
title=			{Introduction to pdf{\TeX{}}}, 
journal=		{TUGboat}, 
volume=		{23},
pages=		{329--334},
number=		{3/4},
url=			{http://www.tug.org/TUGboat/Articles/tb23-3-4/tb75feu.pdf},
year=			2002}
\end{verbatim}
\end{singlespace}
or
\begin{singlespace}\small
\begin{verbatim}
@book{mgbcr04,
author=		{Frank Mittelbach and Michel Goossens and
Johannes Braams and David Carlisle and Chris Rowley},
title=			{The \LaTeX\ Companion},
publisher=		{Addison Wesley Professional},
edition=		{2nd},
address=		{New York},
year=			2004}
\end{verbatim}
\end{singlespace}

For a Web site I would recommend the following
\begin{singlespace}\small
\begin{verbatim}
@misc{brei04,
author = 		{Jon Breitenbucher},
title = 		{{W}ooster related {L}a{T}e{X} files},
url = 			{https://woolatex.spaces.wooster.edu},
howpublished=	{World Wide Web},
year=			2021,
note = 		{Accessed on 09/27/2021}}
\end{verbatim}
\end{singlespace}

You can make a reference by typing \verb|\citet{mgbcr04}| to produce \citet{mgbcr04}. Other forms for citation include \verb|\citep{mgbcr04}| or  \verb|\citeauthor| \verb|{mgbcr04}| to produce \citep{mgbcr04} or \citeauthor{mgbcr04} respectively. You can consult \citet{kd03} or \citet{mgbcr04} to find out how to format entries in the .bib file and what options each reference type has.\footnote{You could also use footnotes if your department called for that.}

Indicies are also relatively easy to create. If I wanted to have Wooster\index{Wooster} show up in the index, I would enter \verb|Wooster\index{Wooster}| in my source file. I could create a subentry for User Services\index{Wooster!User Services} by entering \verb|User Services|\verb|\index{Wooster!User Services}|. A subsubentry for Help Desk\index{Wooster!User Services!Help Desk} would be entered as \verb|\index{Wooster!User| \verb|Services!Help Desk}|.

To create the index, one needs to make sure to uncomment the \verb|\makeindex| command in the \texttt{main.tex} file. One also needs to uncomment the makeidx entry in the \verb|styles/packages.tex| file and then run the Makeindex program. Consult \citet{kd03} or \citet{mgbcr04} for further information.